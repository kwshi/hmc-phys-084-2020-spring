\documentclass{../phys084}

\pset{1}
\author{}
\date{January 29, 2020}

\begin{document}

\section{Matrix math plus}
\begin{exercise}
  Consider the matrix
  \(\begin{bmatrix} 2 & i \\ -i & 2 \end{bmatrix}\).
  \begin{problems}
  \item Find the eigenvalues of this matrix and find an eigenvector
    associated with each eigenvalue.
  \item Could this matrix represent a quantum logic gate?  If not,
    could you multiply it by a scalar (a number) to turn it into a
    matrix that represents a quantum logic gate?  Explain.
  \end{problems}
\end{exercise}

\begin{solution}
\end{solution}

\section{Single-qubit quantum circuit}
\begin{exercise}
  \begin{problems}
  \item Draw a quantum circuit that uses only H, Z, and/or X gates to
    accomplish the following transformation of the qubit states
    \(\ket 0\) and \(\ket 1\):
    \begin{align}
      \ket 0 &\to \frac{1}{\sqrt 2} \ket 0 - \frac{1}{\sqrt 2} \ket 1, \\
      \ket 1 &\to \frac{1}{\sqrt 2} \ket 0 + \frac{1}{\sqrt 2} \ket 1.
    \end{align}
  \item If a qubit in the general state
    \(\ket \psi = \alpha \ket 0 + \beta \ket 1\) is used as the input
    to your quantum circuit, what will the output state be?  Write it
    as a superposition of \(\ket 0\) and \(\ket 1\).
  \item What is the matrix representation of your quantum circuit in
    the \(\set{\ket 0, \ket 1}\) basis?  Find this matrix
    representation in two different ways: first, write down the matrix
    representation from the information given in part (a) about how
    the circuit transforms \(\ket 0\) and \(\ket 1\).  Second, find
    the matrix representation of the circuit by multiplying the matrix
    representations of H, X, and/or Z gates appropriately in sequence.
    (As a qubit progresses through a quantum circuit from left to
    right, matrices multiply the original column vector from right to
    left!)
  \end{problems}
\end{exercise}

\begin{solution}
\end{solution}

\section{Measuring a qubit}

\begin{exercise}
  A qubit is prepared in the quantum state
  \(\ket \psi = \frac{1}{\sqrt 3} \ket 0 + i \sqrt{\frac 2 3} \ket
  1\).
  \begin{problems}
  \item Suppose we perform a measurement to determine whether the
    qubit is in state \(\ket 0\) or \(\ket 1\).  This is called
    \textit{a measurement in the \(\set{\ket 0, \ket 1}\) basis}.
    What is the probability of finding the qubit in state \(\ket 0\)?
  \item Suppose instead, we perform a measurement in the
    \(\set{\ket +, \ket -}\) basis.  In other words, we perform a
    measurement to determine whether the qubit is in state \(\ket +\)
    or \(\ket -\).  What is the probability of finding the qubit in
    state \(\ket +\)?
  \end{problems}
\end{exercise}

\begin{solution}
\end{solution}

\section{Why ``NOT gate'' is a questionable name}
\begin{exercise}
  Suppose we would like to write the matrix representation of the
  so-called ``quantum NOT gate,'' or X gate, in the
  \(\set{\ket +, \ket -}\) basis.  What is that matrix representation?
  Recall that the matrix representation of the X gate in the
  \(\set{\ket 0, \ket 1}\) basis is
  \(\begin{bmatrix} 0 & 1 \\ 1 & 0 \end{bmatrix}\).
\end{exercise}

\begin{solution}
\end{solution}

\end{document}