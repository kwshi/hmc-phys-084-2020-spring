\documentclass{../phys084}

\pset{2}
\author{}
\date{2020 February 5}

\begin{document}

\section{Unitary evolution}

\begin{exercise}
  Recall that a unitary matrix satisfies \(U^\dagger U = I\), where
  \(U^\dagger\) is the complex conjugate transpose of \(U\).
  Likewise, a unitary operator has \(\hat U^\dagger \hat U = \hat I\).

  \begin{problems}
  \item From the facts above, show that a unitary operator must
    preserve the inner product between states.  That is, show that the
    inner product of state \(\ket \psi\) with state \(\ket \phi\) is
    equal to the inner product of state \(\hat U \ket \psi\) with
    state \(\hat U \ket \phi\).  Here the states may be \(n\)-qubit
    states (in which case the unitary operator is an \(n\)-qubit
    operator), so make your argument general.
  \item Show that a \(2 \times 2\) matrix
    \(U = \begin{bmatrix} a & b \\ c & d \end{bmatrix}\) is unitary if
    and only if it satisfies the constraints
    \begin{align*}
      \abs a^2 + \abs c^2 &= 1, \\
      \abs b^2 + \abs d^2 &= 1, \\
      b^*a + d^* c &= 0.
    \end{align*}
    This means an arbitrary \(2 \times 2\) unitary matrix can be
    written in terms of four real numbers \(x\), \(y\), \(w\), and
    \(\Delta\) as follows (you don't have to show this):
    \[
      U =
      \begin{bmatrix}
        e^{i\omega} \cos \frac \Delta 2 & e^{-ix} \sin \frac \Delta 2 \\
        e^{iy} \sin \frac \Delta 2 & e^{i(x+y-w)} \cos \frac \Delta 2
      \end{bmatrix}.
    \]
  \end{problems}
\end{exercise}

\begin{solution}
\end{solution}

\section{Single-qubit gates}

\begin{exercise}
  \begin{problems}
  \item A class of unitary single-qubit operators \(\hat{R_z}(\beta)\)
    is represented in the \(\set*{\ket 0, \ket 1}\) basis by matrices
    \[
      R_z(\beta) =
      \begin{bmatrix}
        e^{-i\beta/2} &  0 \\ 0 & e^{i\beta/2}
      \end{bmatrix}.
    \]
    \textbf{Show} that \(\hat{R_z}(\beta)\) is a rotation by angle
    \(\beta\) around the \(z\) axis on the Bloch sphere.  Remember
    that multiplying a qubit state by an overall phase factor is
    meaningless and can thus be neglected.
  \item Another class of unitary operators \(\hat{R_y}(\Delta)\) is
    represented in the \(\set*{\ket 0, \ket 1}\) basis by matrices
    \[
      R_y(\Delta)=
      \begin{bmatrix}
        \cos \frac \Delta 2 & -\sin \frac \Delta 2 \\
        \sin \frac \Delta 2 & \cos \frac \Delta 2
      \end{bmatrix}.
    \]
    \textbf{Show} that \(\hat{R_y}(\Delta)\) is a rotation by angle
    \(\Delta\) around the ``y'' axis on the Bloch sphere.  Using the
    following reasoning might be helpful: From part (a), we know the
    matrix form of a rotation around the ``z'' axis when expressed in
    the basis of states that point along \(\pm\)z on the Bloch sphere.
    The identical form must also be taken by rotations around the
    ``y'' axis when expressed in the basis of states that point along
    \(\pm y\) on the Bloch sphere.  Show what the action of
    \(\hat{R_y}(\Delta)\) looks like in the
    \(\set*{\frac{1}{\sqrt 2} \prn*{\ket 0 + i \ket 1}, \frac{1}{\sqrt
        2} \prn*{\ket 0 - i \ket 1}}\) basis, to show that it
    represents rotation by \(\Delta\) around the \(y\) axis.
  \item It might be tempting to suggest that a \(\pi\) rotation about
    the ``y'' axis on the Bloch sphere is an X gate, or quantum NOT
    gate.  \textbf{Disprove} that suggestion by by demonstrating an
    input state that does not transform the same way under the two
    operations.  However, an X gate \textit{is} a \(\pi\) rotation
    about some axis on the Bloch sphere.  Taking that fact as given,
    \textbf{specify} what the correct axis is, and justify your
    answer.
  \end{problems}
\end{exercise}

\begin{solution}
\end{solution}

\section{Fidelity of state guesses}

\begin{exercise}
  Consider the following qubit guessing game (name credits to Tommy
  Schneider `20).  Avril chooses a state \(\ket \psi\) randomly and
  prepares a single qubit in that state.  She asks her game partner,
  Boi, to guess the qubit's state.  Whatever guess \(\ket \gamma\) Boi
  makes, Avril then measures the qubit in the
  \(\set*{\ket \gamma, \ket{\gamma_\perp}}\) basis.  Boi wins if the
  qubit is found in \(\ket \gamma\), which should happen with
  probability \(\abs*{\braket \gamma \psi}^2\).  The quantity
  \(\abs*{\braket \gamma \psi}^2\) is also known as the
  \textit{fidelity} \(F\) of Boi's guess.  Avril and Boi sometimes
  play many rounds of this game, in which case Avril picks a new
  random state to start each round.
  \begin{problems}
  \item Suppose Boi employs the following guessing scheme: lacking any
    information about the state, he decides to always guess
    \(\ket 0\).  \textbf{With what probability will Boi win the game?}
    Demonstrate an exact answer by calculating the average guess
    fidelity analytically, \textit{or} demonstrate an approximate
    answer by doing a Monte Carlo simulation of many rounds of the
    game.  If you choose the second option, turn in a (well-commented)
    printout of your code.

    For either choice, you need to think about how Avril would go
    about picking a qubit state at random.  A fair method is to pick a
    random point on the surface of the Bloch sphere, which has some
    values of \(\theta\) (from \(0\) to \(\pi\)) and \(\varphi\) (from
    \(0\) to \(2\pi\)), and then construct the state
    \(\ket \psi = \cos \frac \theta 2 \ket 0 + e^{i\varphi} \sin \frac
    \theta 2 \ket 1\).  Therefore, the probability of selecting a
    state between \(\theta\) and \(\theta + \dif \theta\) and between
    \(\varphi\) and \(\varphi + \dif \varphi\) would be given by the
    surface area of that patch on the Bloch sphere, divided by the
    total surface area of the Bloch sphere:
    \[
      \frac{\sin \theta \dif \theta \dif \varphi}{4 \pi}
    \]
  \item Suppose Avril offers to make the game easier for Boi.  In each
    round, she still chooses a state \(\ket \psi\) at random, but then
    she prepares \textit{two} qubits in the same state \(\ket \psi\).
    She gives one qubit to Boi, who can do whatever he wants with it
    to help him guess the state. Boi gives his guess, and Avril
    measures the second qubit to see if Boi wins, just as in the
    original version of the game.

    Boi decides on the following strategy.  He takes the qubit Avril
    gives him, and measures it in the \(\set*{\ket 0, \ket 1}\) basis.
    If Boi finds the qubit in \(\ket 0\), he guesses that the original
    state was \(\ket 0\), and if he finds the qubit in \(\ket 1\), he
    guesses that the original state was \(\ket 1\).

    \textbf{With what probability will Boi win this revised game?}
    For an individual state \(\ket \psi\), the average fidelity of
    Boi's guessing process will be the probability that Bob measures
    (and guesses) \(\ket 0\) times the fidelity of the \(\ket 0\)
    guess, plus the probability that Boi measures (and guesses)
    \(\ket 1\) times the fidelity of the \(\ket 1\) guess.  To get
    Boi's overall probability of winning, we must further average over
    all possible states \(\ket \psi\) on the Bloch sphere.  As in part
    (a), demonstrate an exact answer by calculating the average guess
    fidelity analytically, \textit{or} demonstrate an approximate
    answer by doing a Monte Carlo simulation of many rounds of the
    game.  If you choose the second option, turn in a (well-commented)
    printout of your code.
  \end{problems}
\end{exercise}

\begin{solution}
\end{solution}

\section{CNOT with change of basis}

\begin{exercise}
  Show that surrounding a CNOT gate with Hadamard transforms on the
  input and output channels is equivalent to reversing its control and
  target qubits:
  \[
    \Qcircuit @C=2em @R=.5em {
      & \gate H & \ctrl 2 & \gate H & \qw &   & & \targ     & \qw \\
      &         &         &         &     & = & &           &     \\
      & \gate H & \targ   & \gate H & \qw &   & & \ctrl{-2} & \qw
    }
  \]
\end{exercise}

\begin{solution}
\end{solution}

\end{document}