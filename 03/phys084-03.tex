\documentclass{../phys084}

\pset{3}
\author{}
\date{2020 February 12}

\begin{document}

\section{The \(n\)-bit Deutsch--Jozsa algorithm}

\begin{exercise}
  The \(n\)-bit Deutsch--Jozsa algorithm determines whether a
  two-valued function of an \(n\)-bit string,
  \(f \colon \set{0, 1}^{\otimes n} \to \set{0, 1}\), is constant or
  balanced.  (For \(n > 1\) most functions are neither constant
  \textit{nor} balanced, but in this scenario we've been assured that
  \(f\) is either one or the other; we only need to decide which.)  An
  oracle F takes an \(n\)-bit input \(x\) (or \(\ket x\)) and a single
  target bit \(y\) (or \(\ket y\)).  It computes \(f(x)\) by taking
  \(y\) to \(y \oplus f(x)\) (or \(\ket y\) to
  \(\ket{y \oplus f(x)}\)) just as in the one-bit algorithm.  In the
  quantum case, \(\ket x\) is an \(n\)-qubit computational basis
  state; \textit{e.g.}, \(\ket 0 \ket 1 \ket 0 \ket 0 \dots \ket 1\)
  to represent \(x = 0100 \dots 1\).

  \begin{problems}
  \item In the classical case, we call the oracle on different \(x\)
    inputs, record the various \(f(x)\) values, and compare them.  In
    the \(n\)-bit classical case, what are the minimum and maximum
    numbers of function calls needed to determine whether the function
    is constant or balanced?

  \item In the next two parts of the problem, you will show that the
    quantum circuit below answers the question in just one function
    call.  \textbf{Explain}, by determining the relevant properties of
    the \((n+1)\)-qubit state through the stages of the computation,
    what \(n\)-qubit measurement result you must get in the circuit
    below if the oracle encodes a constant \(f\).  \label{itm:circuit}

    % \includegraphics[width=300pt]{fig2-1-1.pdf}

  \item Consider the same circuit as in part \ref{itm:circuit}, but
    let the oracle encode a balanced function \(f\).  \textbf{Show}
    that you can \textit{never} get the same \(n\)-qubit measurement
    result as you were guaranteed in part (b).  [Hint: Show that the
    state after the oracle F for a balanced function is orthogonal to
    the state after the oracle F for a constant function.  Unitary
    evolution preserves orthogonality, so now you know that the state
    right before measurement for a balanced function is orthogonal to
    the state right before measurement for a constant function.  Use
    that to show the measurement always yields different results for a
    balanced function vs. a constant one.]

    This circuit is the \(n\)-bit or generalized Deutsch--Jozsa
    algorithm; though David Deutsch and Richard Jozsa contributed
    greatly to its development, the algorithm in its present form was
    actually published by Richard Cleve, Artur Ekert, Chiara
    Macchiavello, and Michele Mosca in 1998. (See, \textit{e.g.},
    \url{arxiv.org/abs/quant-ph/9708016} after you have solved this
    homework problem.)
  \end{problems}
\end{exercise}

\begin{solution}
  \begin{problems}
  \item
  \item
  \item
  \end{problems}
\end{solution}

\section{Creating entangled states}

\begin{exercise}
  One of the entangled ``Bell states'' of two qubits is
  \[
    \ket{\Psi^-}
    = \frac{\ket 0_A \ket 1_B - \ket 1_A \ket 0_B}{\sqrt 2}.
  \]

  Design a quantum logic circuit, using gates we have discussed in
  class, to create this two-qubit state starting from input qubits
  each initialized in the state \(\ket 0\).  Show that your circuit
  produces the state \(\ket{\Psi^-}\) as its output.
\end{exercise}

\begin{solution}
\end{solution}

\section{Measurements on one member of an entangled pair}

\begin{exercise}
  If we have just one qubit in state \(\ket \psi\), and we measure it
  in the \(\set{\ket 0, \ket 1}\) basis, we already know that the
  probability of finding it in state \(\ket 0\) is
  \[
    \abs{\braket 0 \psi}^2
    = \braket \psi 0\braket 0 \psi
    = \bra \psi (\ket 0 \bra 0) \ket \psi.
  \]
  In the last line above, \(\ket 0 \bra 0\) is an \textit{operator}.
  (Pick a basis and consider the matrix representations of all these
  things.  \(\ket 0\) is represented by a column vector, \(\bra 0\) is
  represented by a row vector, and therefore multiplying them in this
  order give a matrix, not a number!)  As discussed in Chapter 4, we
  can call \(\ket 0 \bra 0\) the projection operator \(\hat P_0\) onto
  state \(\ket 0\).  This way of thinking about measurement
  probabilities is helpful when thinking about situations where we
  have several qubits, but we measure only a subset of them.

  Consider the two-qubit state \(\ket{\Psi^-}\), but imagine that we
  perform a measurement on qubit \(A\) only.

  \begin{problems}
  \item Consider measuring qubit \(A\) in the
    \(\set{\ket 0_A, \ket 1_A}\) basis.  The probability of finding
    qubit \(A\) in state \(\ket 0_A\) is given by
    \[
      \bra*{\Psi^-} \prn*{\hat P_{0,A} \otimes \hat I_B} \ket*{\Psi^-};
    \]
    here, \(\hat P_ 0\) is the single-qubit projection operator
    \(\hat P_ 0 = \ket 0 \bra 0\).  The subscript \(A\) indicates that
    it acts on qubit \(A\), and it is tensored with the identity
    operator \(\hat I\) acting on qubit \(B\) since no measurement is
    performed on qubit \(B\).  \textbf{Show} that the probability of
    finding qubit \(A\) in state \(\ket 0_A\) is \(\frac 1 2\).

  \item We may choose to write each qubit's state, not as a
    superposition of \(\ket 0\) and \(\ket 1\), but as a superposition
    of \(\ket \psi\) and \(\ket{\psi_\perp}\), where
    \(\ket 0 = \alpha \ket \psi + \beta \ket{\psi_\perp}\),
    \(\ket 1 = -\beta^* \ket \psi + \alpha^* \ket{\psi_\perp}\), and
    \(\abs \alpha^2 + \abs \beta^2 = 1\).  \textbf{Show} that:
    \[
      \ket{\Psi^-} = \frac
      {\ket \psi_A\ket{\psi_\perp}_B - \ket{\psi_\perp}_A\ket \psi_B}
      {\sqrt 2}.
    \]

  \item If we measure qubit \(A\) in \textit{any}
    \(\set*{\ket \psi_A, \ket{\psi_\perp}_A}\) basis, \textbf{show}
    that the probability of finding qubit \(A\) in the state
    \(\ket \psi_A\) is \(\frac 1 2\).  Thus the state of qubit \(A\)
    by itself, independent of the state of qubit \(B\), is completely
    uncertain and contains no information.  This is the hallmark of
    complete entanglement between two particles.
  \end{problems}
\end{exercise}

\begin{solution}
  \begin{problems}
  \item
  \item
  \item
  \end{problems}
\end{solution}

\section{Controlled \(U\) from \(\CNOT\) and single-qubit unitaries}

\begin{exercise}
  In class, we constructed a general two-qubit controlled \(U\) gate
  from \(\CNOT\) and single-qubit unitaries by first claiming that we
  could write a general one-qubit \(U\) as \(U = e^{i\alpha}AXBXC\).
  Here \(X =
  \begin{bmatrix}
    0 & 1 \\ 1 & 0
  \end{bmatrix}
  \) in the computational basis, and \(A\), \(B\), and \(C\) are
  unitary matrices such that \(ABC=I\).  In this problem, you will
  actually show that a general one-qubit \(U\) can be written this
  way.

  \begin{problems}
  \item Show that \(XR_y (\Delta) X = R_y (-\Delta)\), where
    \(R_y(\Delta) =
    \begin{bmatrix}
      \cos \frac \Delta 2 & -\sin \frac \Delta 2 \\
      \sin \frac \Delta 2 &  \cos \frac \Delta 2
    \end{bmatrix}
    \) represents rotation by angle \(\Delta\) about the \(y\) axis on
    the Bloch sphere.

  \item Show that \(XR_z(\gamma)X = R_z(-\gamma)\), where
    \(R_z(\gamma) =
    \begin{bmatrix}
      e^{-i\gamma/2} & 0 \\ 0 & e^{i\gamma/2}
    \end{bmatrix}
    \) represents rotation by angle \(\gamma\) about the \(z\) axis on
    the Bloch sphere.

  \item Show that
    \begin{align*}
      A &= R_z(\beta') R_y \prn*{\frac \Delta 2}, \\
      B &= R_y\prn*{-\frac \Delta 2} R_z\prn*{-\frac{(\beta+\beta')}{2}}, \\
      C &= R_z\prn*{\frac{\beta-\beta'}{2}}.
    \end{align*}
    satisfy the conditions set out above for expressing an arbitrary
    \(U\).  That is, show that \(ABC = I\) and that
    \(e^{i\alpha}AXBXC\) gives the form we discussed in class for a
    general \(2 \times 2\) unitary matrix.
  \end{problems}
\end{exercise}

\begin{solution}
  \begin{problems}
  \item
  \item
  \item
  \end{problems}
\end{solution}

\end{document}
