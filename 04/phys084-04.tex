\documentclass{../phys084}

\pset{4}
\author{}
\date{2020 February 18}

\DeclareMathOperator{\ising}{ZZ}

\begin{document}
\section{Building a Toffoli gate}
\begin{exercise}
  \begin{problems}
  \item Show that a \(\CNOT\) gate can be constructed from a
    controlled-\(Z\) gate and Hadamards as follows:
    \[
      \Qcircuit @R=1em @C=2em {
        & \ctrl 1 & \qw \\
        & \targ   & \qw
      }
      \qquad = \qquad
      \Qcircuit @R=1em @C=2em {
        & \qw     & \ctrl 1 & \qw     & \qw \\
        & \gate H & \gate Z & \gate H & \qw
      }
    \]

    Thus it would be equally valid to use the controlled-\(Z\) gate as
    the only two-qubit gate in a quantum computing architecture,
    instead of the \(\CNOT\).

  \item Design a quantum circuit using only \(\CNOT\) and
    controlled-\(R_z(\beta)\) gates to perform a
    controlled-controlled-\(Z\) gate on three qubits.  (I won't ask
    you in this problem to make the controlled-\(R_z(\beta)\) gates
    out of \(\CNOT\)s and single-qubit unitaries, since you already
    spent considerable time on the previous problem set showing how
    that sort of thing can be done.)  Demonstrate that your circuit
    performs as desired.  Hint: You will probably want to consult the
    Class 5 slides and/or the ``Chapter 3.5'' notes for ideas on how
    to construct controlled-controlled-\(U\) gates out of simpler
    gates.

  \item Design a quantum circuit using only \(\CNOT\), Hadamard gates,
    and controlled-\(R_z(\beta)\) gates to perform a Toffoli
    (controlled-controlled-\(\NOT\)) gate on three qubits.
    Demonstrate that your circuit performs as desired.
  \end{problems}
\end{exercise}

\begin{solution}
  \begin{problems}
  \item
  \item
  \item
  \end{problems}
\end{solution}

\section{Ising (\(\ising\)) gate}

\begin{exercise}
  In some quantum computing platforms, the easiest two-qubit gate to
  implement is something called the Ising (\(\ising\)) coupling gate.
  In the standard basis, this kind of gate is represented by a matrix
  of the form
  \[
    \ising_\phi =
    \begin{bmatrix}
      e^{i\phi/2} & 0 & 0 & 0 \\
      0 & e^{-i\phi/2} & 0 & 0 \\
      0 & 0 & e^{-i\phi/2} & 0 \\
      0 & 0 & 0 & e^{i\phi/2}
    \end{bmatrix}.
  \]
  \begin{problems}
  \item If \(\phi = \pi\), this gate does not actually involve any
    interaction between the two qubits.  Show that \(\ising_\pi\)
    factors into the form \(U_A \otimes U_B\), so that it just
    represents separate single-qubit gates on the two qubits \(A\) and
    \(B\).

  \item If \(\phi = -\frac \pi 2\), show that the \(\ising_\phi\) gate
    cannot be factored into the form \(U_A \otimes U_B\).  Therefore
    \(\ising_{-\pi/2}\) does represent a nontrivial interaction
    between the two qubits.

  \item Show how a control-Z gate can be accomplished, up to an
    overall phase factor, by a circuit that uses a \(\ising_{-\pi/2}\)
    gate followed by \(R_z(\beta)\)-type gates on the i

  \item Does it matter whether the \(R_z(\beta)\)-type gates on the
    individual qubits come before or after the \(\ising_{-\pi/2}\)
    gate?  Explain.
  \end{problems}
\end{exercise}

\begin{solution}
  \begin{problems}
  \item
  \item
  \item
  \item
  \end{problems}
\end{solution}

\section{Using Hamiltonians to make logic gates}

\begin{exercise}
  Consider the electron spin qubit system we discussed in class and in
  Chapter 4, where \(\ket 0\) is the state with spin up along the
  \(z\) axis, and \(\ket 1\) is the state with spin down along the
  \(z\) axis.  This is also a good model for one type of
  superconducting qubit system, where superconducting loops can be in
  the ``counter-clockwise current'' state (called \(\ket 0\)), the
  ``clockwise current'' state (called \(\ket 1\)), or a quantum
  superposition of the two.

  In describing the physics of either type of quantum computer,
  physicists define an operator \(\hat \sigma_z\) for each qubit, so
  that \(\ket 0\) and \(\ket 1\) are eigenstates of \(\hat \sigma_z\),
  with eigenvalues \(+1\) and \(-1\), respectively.  We also define an
  operator \(\hat \sigma_x\) for each qubit, so that \(\ket +\) and
  \(\ket -\) are eigenstates of \(\hat \sigma_x\) with eigenvalues
  \(+1\) and \(-1\), respectively.  (The names of these operators, and
  the \(+1\) and \(-1\) eigenvalues, come from the spin qubit system.)

  Now think of two qubits A and B, such that the architecture of the
  quantum computer can cause them to interact with each other.  Then
  the Hamiltonian for the two qubits is of the form
  \[
    \hat H_\text{general}
    = h_A \hat \sigma_{z,A} \otimes \hat I_B
    + h_B \hat I_A \otimes \hat \sigma_{z,B}
    + J_{AB} \hat \sigma_{z,A} \otimes \hat \sigma_{z,B}
    + k_A \hat \sigma_{x,A} \otimes \hat I_B
    + k_B \hat I_A \otimes \hat \sigma_{x,B}
  \]
  with \(h_A\), \(h_B\), \(J_{AB}\), \(k_A\), and \(k_B\) as numerical
  parameters that can be adjusted in order to change (or ``tune") the
  Hamiltonian and therefore the time evolution of the qubit states
  according to the Schr\"odinger equation.  In the default or resting
  situation, \(h_A = h_B = J_{AB} = k_A = k_B = 0\), so the states of
  the two qubits do not evolve at all in time.  Parameters are set to
  nonzero values for short time intervals in order to perform logic
  gates.  In the logic gates considered in the rest of this problem,
  \(k_A = k_B = 0\) so the last two terms in \(\hat H_\text{general}\)
  are never present.

  \begin{problems}
  \item Consider the two-qubit computational basis states
    \(\set*{\ket 0_A \ket 0_B, \ket 0_A \ket 1_B, \ket 1_A \ket 0_B,
      \ket 1_A \ket 1_B}\).  What will each of these four states
    become after it has been subjected to a Hamiltonian
    \(\hat H_A = h_A \hat \sigma_{z,A} \otimes \hat I_B\) for a time
    \(t\)?  Write your answer in terms of \(h_A\) and \(t\).

  \item What will each basis state become after it has been subjected
    to a Hamiltonian
    \(\hat H_B = h_B \hat I_B \otimes \hat \sigma_{z,B}\) for time
    \(t\)?  Write your answer in terms of \(h_B\) and \(t\).

  \item What will each basis state become after it has been subjected
    to a Hamiltonian
    \(\hat H_{AB} = J_{AB}\hat \sigma_{z,A} \otimes \hat
    \sigma_{z,B}\) for time \(t\)?  Write your answer in terms of
    \(J_{AB}\) and \(t\).

  \item Determine a way of using the tunable Hamiltonian
    \(\hat H_\text{general}\) to perform (up to an overall phase
    factor) a controlled-\(Z\) gate between qubits A and B.  In other
    words, you will develop a procedure that carries out the
    controlled-\(Z\) gate by setting the parameters \(h_A\), \(h_B\),
    and \(J_{AB}\) to certain values for a time \(t_1\), then setting
    them to new values for a time \(t_2\), and so on for as many steps
    as your procedure requires.  Specify the parameter values and time
    intervals at each step of your procedure.  Show that your
    procedure does indeed carry out a controlled-\(Z\) gate, up to an
    overall phase factor.
  \end{problems}
\end{exercise}

\begin{solution}
  \begin{problems}
  \item
  \item
  \item
  \item
  \end{problems}
\end{solution}

\section{Inversion about the mean}

\begin{exercise}
  For a system of \(n\) qubits, we denote each of the \(2^n\)
  computational basis states by \(\ket x\), where \(x\) ranges from
  \(0\) to \(2^n-1\).  Let \(\ket \Psi\) denote the equal
  superposition of all \(2^n\) basis states,
  \[
    \ket \Psi = \frac{1}{2^{n/2}} \sum^{2^n-1}_{x=0} \ket x.
  \]
  The operator \(\prn*{2 \ket \Psi \bra \Psi - \hat I}\) comes up in
  Grover's algorithm.  It is sometimes called \textit{inversion about
    the mean}; in this problem we will see one motivation for the
  name.

  Consider a general \(n\)-qubit state
  \(\sum^{2^n-1}_{x=0} \alpha_x \ket x\).  This state can be rewritten
  as
  \[
    \sum_{x=0}^{2^n-1}
    (\abr \alpha + (\alpha_x - \abr \alpha)) \ket x
  \]
  where \(\abr \alpha = \frac{1}{2^n} \sum_x \alpha_x\) is the mean
  value of the \(\alpha_x\) coefficients.  Show that
  \[
    \prn*{2 \ketbra \Psi \Psi - \hat I}
    \prn*{\sum_x \alpha_x \ket x}
    = \sum_{x=0}^{2^n-1}
    (\abr \alpha - (\alpha_x - \abr \alpha)) \ket x.
  \]
  Thus the operator flips all the coefficients around the mean
  coefficient value.
\end{exercise}

\begin{solution}

\end{solution}

\end{document}