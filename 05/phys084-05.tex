\documentclass{../phys084}

\pset{5}
\author{}
\date{2020 February 25}

\begin{document}

\section{Eigenstates and Eigenvalues of Modular Multiplication}

\begin{exercise}
  Consider the unitary operator \(\hat U\) which acts on \(n\)-qubit
  computational basis states \(\ket w\) according to the rule
  \(\hat U \ket w = \ket{a \cdot w \bmod N}\) for integers \(N\) and
  \(a<N\).  Let \(r\) be the smallest counting number such that
  \(a^r = 1 \pmod N\).  (This is the same as saying \(r\) is the order
  of \(a\) modulo \(N\).)  In the factoring scenario, \(r\) is unknown
  and the goal of our quantum algorithm is to estimate its value.

  \begin{problems}
  \item Show that the states \(\ket{u_h}\) defined below are
    eigenstates of \(\hat U\) for integer values of \(h\).  Find the
    eigenvalue associated with \(\ket{u_h}\).

    \[
      \ket{u_h} = \frac{1}{\sqrt r} \sum_{k=0}^{r-1}
      \exp \prn*{\frac{-i2\pi h k}{r}} \ket*{a^k \bmod N}.
    \]

  \item Show that
    \[
      \frac{1}{\sqrt r} \sum_{h=0}^{r-1} \ket{u_h}
      = \ket{\mathbf 1} = \ket{0 \dots 01}.
    \]
    [These facts explain how the order-finding algorithm can be
    thought of as a phase-estimation algorithm for \(\hat U\).  The
    order-finding ``output-register'' qubits initialized in
    \(\ket{\mathbf 1} = \ket{0 \dots 01}\) form the \(n\)-qubit
    reference register for phase estimation; part (b) shows that being
    initialized in \(\ket{0...01}\) is the same as being initialized
    in a superposition of various eigenvectors of \(\hat U\).  The
    order-finding ``input'' register forms the \(t\)-qubit phase
    register for phase estimation (it just so happens here that
    \(t=n\) so both registers have the same number of qubits).  Both
    algorithms call for it to be initialized in the equal
    superposition of all basis states.  The measurement at the end of
    the order-finding algorithm is highly likely to give the integer
    closest to \(\frac{2^n \cdot h}{r}\) for some integer \(h\).  Part
    (a) shows how this is explained in terms of phase estimation: the
    final measurement is highly likely to give an integer
    approximation of \(2^n\) times \(\frac{\theta}{2\pi}\), where
    \(e^{i\theta}\) is some eigenvalue of \(\hat U\) belonging to an
    eigenvector \(\ket{u_h}\).]
  \end{problems}
\end{exercise}

\begin{solution}
  \begin{problems}
  \item
  \item
  \end{problems}
\end{solution}

\section{2-Qubit Quantum Fourier Transform}

\begin{exercise}
  \begin{problems}
  \item Design and draw out a circuit that accomplishes the two-qubit
    quantum Fourier transform using only Hadamard, controlled-\(S\),
    and/or \(\CNOT\) gates.  (\(S = e^{i\pi/4}R_z(\frac \pi 2) =
    \begin{bmatrix}
      1 & 0 \\ 0 & e^{i\pi/2}
    \end{bmatrix}
    \) in the computational basis.)

  \item Write the matrix representation of the two-qubit quantum
    Fourier transform in the standard two-qubit computational basis.
  \end{problems}
\end{exercise}

\begin{solution}
  \begin{problems}
  \item
  \item
  \end{problems}
\end{solution}

\end{document}