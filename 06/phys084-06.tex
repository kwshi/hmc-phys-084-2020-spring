\documentclass{../phys084}

\pset{6}
\author{}
\date{2020 March 11}

\begin{document}

\section{Alternate View of the 3-Qubit Bit-Flip Code}

\begin{exercise}
  We saw in class how to correct single bit-flip errors with the 3-qubit
  bit-flip code.  In that version of the error-correction protocol, we
  measured our two ancilla qubits in the \(\set{\ket 0, \ket 1}\) basis,
  giving four possible measurement results.  Each measurement result
  instructed us to perform a particular error-correcting operation on
  the three code qubits.
  \begin{problems}
  \item Draw an error-correction circuit that performs the same 3-qubit
    bit-flip correction without explicit measurement of the ancilla
    qubits.  (Instead of measurement of the ancilla qubits and
    conditional operations on the code qubits, try controlled-\(U\)
    and/or controlled-controlled-\(U\) gates with the ancilla qubits as
    controls and the code qubits as targets.)  Your circuit can use any
    one-, two-, or three-qubit gates we have learned about in class,
    including \(\CNOT\) and Toffoli gates.
  \item At an arbitrary stage of a long quantum computation, the three
    code qubits may be entangled with an external set of qubits, in a
    state such as
    \(\alpha \ket{000} \ket \psi_\text{ext} +
    \beta\ket{111}\ket{\psi_{\perp}}_\text{ext}\).  Convince yourself
    (and me!)  that our error correction procedure protects even an
    entangled state like this against single bit-flip errors on the
    three code qubits.
  \end{problems}
\end{exercise}

\begin{solution}
  \begin{problems}
  \item
  \item
  \end{problems}
\end{solution}

\section{Smallest Code for the Job}

\begin{exercise}
  Suppose we need to encode a general single-qubit state
  \(\ket \psi = \alpha\ket 0 + \beta \ket 1\) in the states of \(n\)
  physical qubits.  Without focusing on a particular error process or
  a particular code, we can make some general statements about the
  nature of the encoding situation.  Some \(n\)-qubit physical state
  \(\ket{0_L}\) will represent \(\ket 0\), and some orthogonal
  \(n\)-qubit physical state \(\ket{1_L}\) will represent \(\ket 1\).
  Thus the initial encoded states \(\alpha\ket{0_L} + \beta\ket{1_L}\)
  can be found in a two-dimensional subspace---the ``code space''---of
  the overall \(2^n\)-dimensional state space for \(n\) qubits.  Each
  particular error process maps the code space to some other
  two-dimensional subspace.
  \begin{problems}
  \item A bit-flip code diagnoses and corrects a bit-flip error in any
    one of the \(n\) qubits.  It also diagnoses the ``no-error'' state
    and corrects \textit{it} by leaving it alone.  To permit this kind
    of reliable diagnosis of \((n+1)\) different conditions, requiring
    \((n+1)\) different correction protocols, it is necessary that the
    code space be mapped to \((n+1)\) mutually orthogonal subspaces by
    the \((n+1)\) different error or no-error processes.  Show that
    this reasoning leads to the requirement \(2^n \geq 2(n+1)\) and
    thus \(n \geq 3\).  Thus the 3-qubit bit-flip code is the smallest
    possible code that can correct bit-flip errors.
  \item Suppose we require an \(n\)-qubit code that can correct a bit
    flip, a phase flip, or a combined bit-and-phase flip on any one of
    the \(n\) qubits.  Develop a condition for the allowed values of
    \(n\), and use it to show that \(n \geq 5\) for such a code.
  \end{problems}
\end{exercise}

\begin{solution}
  \begin{problems}
  \item
  \item
  \end{problems}
\end{solution}

\section{9-Qubit Shor Encoding Circuit}

\begin{exercise}
  The 9-qubit Shor code encodes \(\ket 0\) and \(\ket 1\) as
  \begin{align*}
    \ket{0_L} &= \frac
    {(\ket{000}+\ket{111})(\ket{000}+\ket{111})(\ket{000}+\ket{111})}
    {2 \sqrt 2}, \\
    \ket{1_L} &= \frac
    {(\ket{000}-\ket{111})(\ket{000}-\ket{111})(\ket{000}-\ket{111})}
    {2\sqrt 2}.
  \end{align*}
  This code is a three-qubit phase-flip code, with a three-qubit
  bit-flip code nested inside it.

  Design a circuit that begins with a single qubit in state
  \(\ket \psi = \alpha\ket 0 + \beta\ket 1\) and eight qubits in state
  \(\ket 0\), and produces all nine qubits in the encoded state
  \(\alpha\ket{0_L} + \beta\ket{1_L}\).  Explain or demonstrate how
  your circuit produces the desired output.  [Hint: Draw inspiration
  from the encoding circuits for the phase-flip code and the bit-flip
  code individually.]
\end{exercise}

\begin{solution}

\end{solution}

\end{document}
