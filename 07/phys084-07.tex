\documentclass{../phys084}

\pset{7}
\author{}
\date{2020 April Fools}

\begin{document}

\section{Just a \textit{Small} Error}

\begin{exercise}
  Suppose we are using the Steane to encode a logical qubit in the
  states of seven physical qubits, thus detecting and correcting any
  accidental \(\hat X\) operation, accidental \(\hat Z\) operation, or
  accidental \(\hat X \hat Z\) operation on a single physical qubit.
  Instead of one of these errors, suppose the first physical qubit has
  undergone an accidental \(\hat R_y \prn*{\frac \pi 3}\).  Here
  \(\hat R_y \prn*{\frac \pi 3}\) is defined as usual, so that its
  matrix representation in the computational basis is
  \[
    R_y \prn*{\frac \pi 3} =
    \begin{bmatrix}
      \cos \frac \pi 6 & -\sin \frac \pi 6 \\
      \sin \frac \pi 6 & \cos \frac \pi 6
    \end{bmatrix}.
  \]
  \begin{itemize}
  \item What fraction of the time will the Steane code syndrome
    measurement circuit diagnose no error?
  \item What fraction of the time will it diagnose an accidental
    \(\hat X\) the first qubit?
  \item What fraction of the time will it diagnose an accidental
    \(\hat Z\) on the first qubit?
  \item What fraction of the time will it diagnose an accidental
    \(\hat X \hat Z\) on the first qubit?
  \item Does the Steane code end up reliably correcting this
    \(\hat R_y \prn*{\frac \pi 3}\) error?
  \end{itemize}
\end{exercise}

\begin{solution}

\end{solution}

\section{7-Qubit Steane Code Logical S and T Gates}

\begin{exercise}
  Recall that the \(S\) gate is represented in the
  \(\{\ket 0,\ket 1\}\) basis by the matrix
  \[
    S =
    \begin{bmatrix} 1 & 0 \\ 0 & i \end{bmatrix} =
    \begin{bmatrix} 1 & 0 \\ 0 & e^{i \pi/2} \end{bmatrix},
  \]
  and that the \(T\) gate is represented in the \(\{\ket 0,\ket 1\}\)
  basis by the matrix
  \[
    T = \begin{bmatrix} 1 & 0 \\ 0 & e^{i \pi/4} \end{bmatrix}.
  \]

  \begin{problems}
  \item A logical \(S\) gate (called a \(S_L\) gate) on an encoded
    qubit state is a gate (really accomplished by a multiple-gate
    circuit) that takes \(\ket{0_L}\) to \(\ket{0_L}\) and takes
    \(\ket{1_L}\) to \(i \ket{1_L}\).  For the 7-qubit Steane code,
    show that performing an \(S\) gate on each individual qubit does
    not accomplish the logical \(S\) operation.
  \item Show that performing \(S\) and \(Z\) on each qubit
    \textit{does} accomplish the logical \(S\) operation.
  \item Show that there is \textbf{no} operation represented by a
    matrix of the form
    \[
      \begin{bmatrix} 1 & 0 \\ 0 & e^{i\theta} \end{bmatrix}
    \]
    that can be applied to each individual qubit to accomplish the
    logical \(T\) operation.
  \end{problems}
\end{exercise}

\begin{solution}

\end{solution}

\section{Nonlocality of Coded States}

\begin{exercise}
  The two full QEC codes we have studied in class are designed to
  combat local error processes, \textit{i.e.}, errors that occur on
  individual qubits in an uncorrelated way.  We have claimed that our
  codes protect against these local errors by encoding information
  nonlocally, in the correlations between the states of different
  qubits rather than in the state of any individual qubit by itself.
  In this problem, you will learn that the code words (\(\ket{0_L}\)
  and \(\ket{1_L}\)) for both the 9-qubit Shor code and the 7-qubit
  Steane code are truly nonlocal, entangled states of the individual
  qubits.

  Consider a state \(\ket{\mathbf \Psi}\) of \(n\) qubits that,
  written with the state of the first qubit separated from the other
  \(n-1\), appears as follows:
  \[
    \ket{\mathbf \Psi} =
    \frac{1}{\sqrt 2} \prn*{\ket 0 \ket \phi + \ket 1 \ket \chi},
  \]
  with \(\braket \chi \phi = 0\).  The logic of Problem 3.3 can be
  applied to this \(n\)-qubit state to show that the lone qubit has
  probability \(\frac 1 2\) to be found in any, arbitrary state
  \(\ket \psi = \alpha \ket 0 + \beta \ket 1\).  Thus if an
  \(n\)-qubit state can be written in this form, the state of the lone
  qubit on its own is completely undefined.
  \begin{problems}
  \item For the encoded states \(\ket{0_L}\) and \(\ket{1_L}\) of the
    9-qubit Shor code, show that the state of each qubit on its own is
    completely undefined.
  \item For the encoded states \(\ket{0_L}\) and \(\ket{1_L}\) of the
    7-qubit Steane code, show that the state of each qubit on its own
    is completely undefined.  You may want to use the fact that each
    computational basis state appearing in the encoded state
    \(\ket{0_L}\) has an even number of \(1\)'s, while each
    computational basis state appearing in the encoded state
    \(\ket{1_L}\) has an odd number of \(1\)'s.
  \end{problems}
\end{exercise}

\begin{solution}

\end{solution}

\end{document}
