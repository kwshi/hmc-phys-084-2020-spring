\documentclass{../phys084}

\pset 8
\author {}
\date {2020 April 8}

\begin {document}

\section {Single-Qubit Density Matrix}

\begin {exercise}
  Consider an arbitrary density matrix \(\rho\) for a single qubit (in
  a pure state \textit{or} a mixed state).  This density matrix
  represents, in the computational basis, a density operator
  \(\hat \rho\) which can be written as
  \(\hat \rho = \sum_i p_i \proj {\psi_i}\) for some ensemble of
  states \(\set {\ket {\psi_i}}\) and probabilities \(p_i\).  This
  implies various properties of the density matrix \(\rho\),
  including: \(\rho\) is Hermitian, \(\rho\) is diagonalizable with
  positive eigenvalues, \(\tr \rho = 1\), and \(\tr \rho^2 \le 1\).

  \begin {problems}
  \item We define matrices
    \begin {align*}
      I &= \begin {bmatrix} 1 & 0 \\ 0 & 1 \end {bmatrix}, &
      \sigma_x &= \begin {bmatrix} 0 & 1 \\ 1 & 0 \end {bmatrix}, \\
      \sigma_y &= \begin {bmatrix} 0 & -i \\ i & 0 \end {bmatrix}, &
      \sigma_z &= \begin {bmatrix} 1 & 0 \\ 0 & -1 \end {bmatrix}.
    \end {align*}
    (The first matrix is, of course, the \(2 \times 2\) identity
    matrix, and physicists will recognize the other three as the Pauli
    matrices.)  Show that an arbitrary single-qubit density matrix can
    be written:
    \[
      \rho = \frac {I + \vec r \cdot \vec \sigma} 2
      = \frac {I + r_x \sigma_x + r_y \sigma_y + r_z \sigma_z} 2,
    \]
    where \(\vec r\) is a real, three-dimensional vector with length
    less than or equal to 1.  This vector is called the \textit{Bloch
      vector} for the single-qubit state represented by \(\rho\).
  \item Show that the Bloch vector has length 1 if and only if
    \(\rho\) represents a pure state.  Thus pure states have Bloch
    vectors on the surface of the Bloch sphere (unit sphere),
    consistent with our previous understanding; mixed states have
    Bloch vectors on the \textit {interior} of the Bloch sphere.
  \end {problems}
\end {exercise}

\begin {solution}
  \begin {problems}
  \item
  \item
  \end {problems}
\end {solution}

\section {Bloch Vectors and the Bloch Sphere}

\begin {exercise}
  \begin {problems}
  \item Consider a single-qubit pure state
    \(\ket \psi = \cos \frac \theta 2 \ket 0 + e^{i \varphi} \sin
    \frac \theta 2 \ket 1\).  Find the Bloch vector for this state
    according to the prescription of Problem 8.1, and show that it is
    the same as the vector on the Bloch sphere that we associated with
    this pure state in Chapter 2.
  \item Consider the single-qubit density operator
    \(\hat \rho = \frac 1 2 \ket \psi \bra \psi + \frac 1 2 \proj
    {\psi_{\perp}}\).  We saw in class that, independent of the
    specific state \(\ket \psi\), this density operator is always
    equal to \(\frac 1 2 \proj 0 + \frac 1 2 \proj 1\).  What is the
    density matrix representing this density operator?  Show that the
    Bloch vector for this mixed state is the vector of length zero.
  \end {problems}
\end {exercise}

\begin {solution}
  \begin {problems}
  \item
  \item
  \end {problems}
\end {solution}

\section {Entangled or Not?}

\begin {exercise}
  For each of the following two-qubit pure states,
  \begin {enumerate*} [label=(\roman*)]
  \item give the full density matrix in the computational basis
    \(\{\ket{00},\ket{01},\ket{10},\ket{11}\}\),
  \item perform the
    partial trace over qubit A to find the reduced density matrix
    \(\rho_B\) in the \(\{\ket 0,\ket 1\}\) basis, and
  \item use the
    reduced density matrix to determine whether the two qubits were
    (at least partly) entangled with each other in the original state.
  \end {enumerate*}
  \begin {problems}
  \item \(\frac {\ket {00} + \ket {11}} {\sqrt 2}\)
  \item \(\frac {\ket {00} + \ket {01} + \ket {10}} {\sqrt 3}\)
  \item \(\frac {\ket {00} + \ket {01} + \ket {10} + \ket {11}} 2\)
  \end {problems}
\end {exercise}

\begin {solution}
  \begin {problems}
  \item
  \item
  \item
  \end {problems}
\end {solution}

\section{Quantum Dense Coding with Qutrits}

\begin {exercise}
  In this course, we have focused almost entirely on qubits---quantum
  particles with two-dimensional state spaces, spanned by the
  orthogonal states \(\set {\ket 0, \ket 1}\).  In this problem, let
  us instead use quantum particles where each particle has three
  mutually distinguishable (and thus orthonormal) states
  \(\set {\ket 0, \ket 1, \ket 2}\).  A general pure state of a
  \textit{qutrit} like this is thus written
  \(\alpha \ket 0 + \beta \ket 1 + \gamma \ket 2\).

  Suppose Alice and Bob share a pair of qutrits in the entangled state
  \[
    \ket {\Phi^0_0}_{AB}
    = \frac 1 {\sqrt 3}
    \prn* {\ket 0_A \ket 0_B + \ket 1_A \ket 1_B + \ket 2_A \ket 2_B}.
  \]

  \begin {problems}
  \item Verify that this state is entangled by writing the density
    matrix in the
    \(\set {\ket {00}, \ket {01}, \ket {02}, \ket {10}, \ket {11},
      \ket {12}, \ket {20}, \ket {21}, \ket {22}}\) basis and
    performing the partial trace over qutrit \(A\) to find the reduced
    density matrix for qutrit \(B\).
  \item
    \label{itm:nine}
    Alice intends to send Bob one of nine pre-arranged messages by
    performing an operation on her qutrit to transform the entangled
    state into one of the nine states listed below.  She will then
    send her qutrit to Bob so he can measure both qutrits together and
    determine which message Alice was sending.  Verify that the nine
    \textit{qutrit Bell states} listed below are mutually orthogonal,
    so that Bob can in principle perform a measurement on the two
    particles to distinguish these nine states from each other
    unambiguously.  \newcommand {\kk} [2] {\ket {#1}_A \ket {#2}_B}
    \begin {align*}
      \ket {\Phi^0_0}_{AB} &= \frac 1 {\sqrt 3}
      (\kk 0 0 + \kk 1 1 + \kk 2 2) \\
      \ket {\Phi^1_0}_{AB} &= \frac 1 {\sqrt 3}
      (\kk 0 0 + e^{i2\pi/3} \kk 1 1 + e^{i4\pi/3} \kk 2 2) \\
      \ket {\Phi^2_0}_{AB} &= \frac 1 {\sqrt 3}
      (\kk 0 0 + e^{i4\pi/3} \kk 1 1 + e^{i2\pi/3} \kk 2 2) \\
      \ket {\Phi^0_1}_{AB} &= \frac 1 {\sqrt 3}
      (\kk 0 1 + \kk 1 2 + \kk 2 0) \\
      \ket {\Phi^1_1}_{AB} &= \frac 1 {\sqrt 3}
      (\kk 0 1 + e^{i2\pi/3} \kk 1 2 + e^{i4\pi/3} \kk 2 0) \\
      \ket {\Phi^2_1}_{AB} &= \frac 1 {\sqrt 3}
      (\kk 0 1 + e^{i4\pi/3} \kk 1 2 + e^{i2\pi/3} \kk 2 0) \\
      \ket {\Phi^0_2}_{AB} &= \frac 1 {\sqrt 3}
      (\kk 0 2 + \kk 1 0 + \kk 2 1) \\
      \ket {\Phi^1_2}_{AB} &= \frac 1 {\sqrt 3}
      (\kk 0 2 + e^{i2\pi/3} \kk 1 0 + e^{i4\pi/3} \kk 2 1) \\
      \ket {\Phi^2_2}_{AB} &= \frac 1 {\sqrt 3}
      (\kk 0 2 + e^{i4\pi/3} \kk 1 0 + e^{i2\pi/3} \kk 2 1).
    \end {align*}
  \item Suppose Alice has access to a phase-shifting operator
    \(\hat P\) and a value-shifting operator \(\hat V\) that she can
    apply to her qutrit.  These are represented in the
    \(\set {\ket 0, \ket 1, \ket 2}\) basis as:
    \begin {align*}
      \hat P \rightarrow P &=
      \begin {bmatrix}
        1 & 0 & 0 \\
        0 & e^{i2\pi/3} & 0 \\
        0 & 0 & e^{i4\pi/3}
      \end {bmatrix}, \\
      \hat V \rightarrow V &=
      \begin {bmatrix}
        0 & 1 & 0 \\
        0 & 0 & 1 \\
        1 & 0 & 0
      \end {bmatrix}.
    \end {align*}
    Specify the nine operations by Alice, composed of applications of
    \(\hat V\) and/or \(\hat P\) to her qutrit in various
    combinations, that will turn the initial entangled state into each
    of the nine entangled states listed in part \ref{itm:nine}.
  \end {problems}
\end {exercise}

\begin {solution}
  \begin {problems}
  \item
  \item
  \item
  \end {problems}
\end {solution}

\end {document}
